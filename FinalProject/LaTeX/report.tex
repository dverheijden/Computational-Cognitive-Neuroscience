\documentclass{article}
\usepackage{a4wide}
\usepackage[utf8]{inputenc}
\usepackage[table,xcdraw]{xcolor}
\usepackage{ltablex} % table stretch multiple pages
\usepackage{booktabs} % Fancy table lines
\usepackage[numbers]{natbib}
\usepackage{graphicx}
\usepackage{enumerate}
\usepackage{array}
\usepackage[]{mathtools}
\usepackage{textcomp}
\usepackage{gensymb}
\usepackage{float}
\usepackage{amsmath}
\usepackage{dsfont}
\usepackage{csquotes}
\newcolumntype{b}{X}
\newcolumntype{s}{>{\hsize=.5\hsize}X}
\usepackage{pdfpages}
\usepackage{hyperref} % For links in documents and to the web

\newcommand{\q}[1]{``#1''} %matching apostrophes

\title{Final Project \\ Progressive Neural Networks}
\author{Dennis Verheijden s4455770 \and Joost Besseling s4796799}

\begin{document}
\maketitle


\section*{Introduction}

For the Final Project, we decided to implement Progressive Neural Networks, as introduced by Google. We planned on running them on Pong, using the Atari Gym. Unfortunately, this was more work than we expected. That is why, after some consideration, we decided to change the scope of our project and simply try to run a network on the Atari Gym.
\section{Goal}

We wanted to implement a Neural Network on the atari gym. We wanted to do this to figure out how to use Neural Networks on real time data. A lot of applications in the real world, like self driving cars, will acquire us to run the network in real time.

\section{Implementation}

To implement our network, we used the Chainer package, as required by the course. We used the following network structure.

$\underbrace{\text{A representation of the network here}}$

\

\end{document}